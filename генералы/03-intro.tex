% Введение
% Центрируем заголовок и делаем его капсом
\begin{center}
    \MakeUppercase{\large Введение}
\end{center}
\addcontentsline{toc}{section}{ВВЕДЕНИЕ} % Добавляем в оглавление

Современные распределённые системы играют центральную роль в различных областях информационных технологий: от баз данных и систем управления до облачных вычислений и криптовалют. Их широкое применение обуславливает высокие требования к надёжности, отказоустойчивости и согласованности между компонентами. Однако распределённая природа таких систем сопряжена с рядом фундаментальных проблем, одной из которых является задача достижения консенсуса в условиях ненадёжности некоторых узлов.

Проблема надёжного согласования в распределённых системах впервые была формализована в виде задачи византийских генералов (ЗВГ). Эта задача иллюстрирует ситуацию, в которой несколько сторон (узлов) пытаются прийти к единому решению, несмотря на то, что часть из них может вести себя некорректно — например, из-за сбоев, атак злоумышленников или неправильной конфигурации. Сам термин отсылает к историческим аллюзиям о коммуникации между военачальниками, пытающимися согласовать действия в условиях возможного предательства, что подчёркивает сложность задачи.

Развитие ЗВГ началось с работ Лесли Лампорта, Роберта Шостака и Маршалла Пиза в 1982 году, где была представлена теоретическая основа проблемы и предложены базовые алгоритмы её решения. С тех пор задача стала основой для разработки множества алгоритмов и протоколов, включая PBFT (Practical Byzantine Fault Tolerance), Raft и популярные механизмы консенсуса в блокчейне, такие как Proof-of-Work и Proof-of-Stake.

Задача ЗВГ имеет важное практическое значение. Она лежит в основе отказоустойчивости критически важных систем, таких как банковские платформы, авиационные и медицинские системы управления, а также распределённые реестры. Её решение позволяет минимизировать риски, связанные с несанкционированным поведением участников системы, и повысить уровень доверия в децентрализованных сетях.

Целью данной научно-исследовательской работы является исследование методов решения задачи византийских генералов.

\vspace{0.25cm}
В рамках работы были поставлены следующие задачи:

\begin{enumerate}

\item Провести анализ предметной области: дать основные определения, изучить исторические аспекты развития проблемы;

\item Формализовать задачу византийских генералов;

\item Перечислить существующие методы решения проблемы;

\item Разработать критерии сравнения алгоритмов;

\item Провести сравнительный анализ методов решения на основе сформулированных критериев.

\end{enumerate}

\newpage