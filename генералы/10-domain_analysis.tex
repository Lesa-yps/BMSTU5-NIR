% Аналитиз предметной области
\section{Аналитиз предметной области}

\hspace{1.25cm}
В данном разделе будут представлены основные определения, связанные с методами решения задачи византийских генералов, исторические вехи развития проблемы.

\subsection{Основные определения}

\hspace{1.25cm}
Распределённая система -- совокупность взаимодействующих узлов, работающих совместно для достижения общей цели.

Отказоустойчивость -- способность системы продолжать функционировать корректно при наличии сбоев.

Консенсус -- соглашение между узлами системы относительно общего состояния или принятого решения. Основное требование консенсуса -- выполнение двух свойств:

\begin{enumerate}

\item Согласованность (Agreement) -- все честные узлы принимают одинаковое решение.

\item Достоверность (Validity) -- если командующий честен, принятое решение совпадает с его предложением.
\end{enumerate}


\subsection{История развития проблемы византийских генералов}

\subsubsection*{\underline{Формулировка проблемы}}

\hspace{1.25cm} Проблема византийских генералов, концепция, имеющая ключевое значение в области информатики и распределённых систем, была впервые представлена в 1982 году Лесли Лэмпортом, Робертом Шостаком и Маршаллом Пизом~\cite{lamport}. В своей работе они описали сценарий, в котором несколько генералов должны согласовать стратегию атаки или отступления, несмотря на наличие предателей, которые могут распространять ложную информацию. Это постановка задачи является основой для разработки алгоритмов консенсуса в распределённых системах, где некоторые узлы могут быть ненадёжными или действовать злонамеренно.

Исследование получило широкую поддержку со стороны таких престижных организаций, как НАСА, Командование систем противоракетной обороны и Исследовательское бюро армии. Эти исследования подчеркнули значимость проблемы не только в военной связи, но и в широком спектре компьютерных систем. Вопрос достижения согласия между различными компонентами системы стал важной частью теории отказоустойчивости распределённых систем. Также проблема была рассмотрена в контексте новых подходов к дистрибуции вычислительных процессов и в применении для решения задач в распределённых вычислительных системах.~\cite{plisio}

\subsubsection*{\underline{Ранние подходы к решению}}

\hspace{1.25cm} После публикации работы Лэмпорта, Шостака и Пиза начались активные исследования по решению задачи. Одним из важнейших результатов стало доказательство, что для обеспечения консенсуса в византийской модели необходимо минимум \(3f + 1\) узлов в системе, где \(f\) — количество потенциально ненадёжных узлов. Это открытие стало основой для ограничения разработок в области распределённых систем.

В ответ на вызовы, стоящие перед распределёнными системами, Лэмпорт предложил алгоритм, использующий гарантированную доставку сообщений для обеспечения согласованности. Однако такие методы оказались крайне затратными по вычислительным ресурсам и применимыми только для синхронных систем с заранее известными задержками.

\subsubsection*{\underline{Развитие BFT-протоколов}}

\hspace{1.25cm} В 1990-х годах были разработаны более практичные алгоритмы византийской отказоустойчивости (BFT). Наибольшее внимание привлёк алгоритм \textit{PBFT (Practical Byzantine Fault Tolerance)}, предложенный Барбарой Лисков и Мигелем Кастро в 1999 году. Этот протокол был ориентирован на асинхронные системы с неизвестными задержками и позволил значительно повысить отказоустойчивость распределённых сетей, однако он столкнулся с проблемами масштабируемости.

\subsubsection*{\underline{Проблема в эпоху блокчейна}}

\hspace{1.25cm} В 2008 году Сатоши Накамото предложил революционное решение византийской проблемы в децентрализованных сетях. Он представил алгоритм доказательства работы (\textit{Proof-of-Work, PoW}), который стал основой для первой криптовалюты — \textit{Bitcoin}. Этот алгоритм позволил достичь консенсуса в среде, где узлы не доверяют друг другу и могут быть ненадёжными.~\cite{plisio}.

Накамото успешно объединил достижения из области децентрализованных технологий, предложив решение, которое устраняет необходимость в доверенных третьих сторонах и делает систему безопасной и масштабируемой.

\subsubsection*{\underline{Насущность в современности}}

\hspace{1.25cm} Проблема византийских генералов остаётся важной задачей в области распределённых технологий. Современные исследования сосредоточены на разработке более производительных и энергоэффективных протоколов консенсуса, способных работать в условиях больших сетей и высокой нагрузки.

Алгоритмы византийского консенсуса нашли широкое применение в различных областях, включая блокчейн, распределённые базы данных, системы интернета вещей (\textit{IoT}) и кибербезопасность. Эти алгоритмы продолжают служить основой для новых децентрализованных технологий, обеспечивая надёжность, безопасность и масштабируемость современных систем.

\newpage