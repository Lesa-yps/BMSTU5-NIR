% Реферат

\pagenumbering{arabic} % Включить нумерацию страниц в арабском формате 
\setcounter{page}{2} % Установить номер страницы 2

\begin{center}
    \MakeUppercase{\large Реферат}
\end{center}
\addcontentsline{toc}{section}{РЕФЕРАТ} % Добавляем в оглавление

Научно-исследовательская работа содержит \pageref*{LastPage} с., \totalfigures\ рис., \totaltables\ табл., 10 ист., 1 прил.

\underline{Ключевые слова:}

Византийские генералы, распределенные системы, консенсус, алгоритмы, отказоустойчивость.


\underline{Объект исследования:}

Решение задачи византийских генералов — классической проблемы в распределенных системах, связанной с достижением консенсуса в условиях ненадежных узлов.


\underline{Результаты работы:}

\begin{enumerate}

\item Изучена предметная область, определены основные понятия и рассмотрены исторические аспекты проблемы; 

\item Сформулирована и формализована задача византийских генералов; 

\item Описаны существующие методы решения проблемы; 

\item Разработаны критерии для сравнения и оценки алгоритмов; 

\item Проведён анализ различных методов с использованием предложенных критериев и сделаны выводы.

\end{enumerate}