% Методы решения задачи
\section{Методы решения задачи византийских генералов}

\subsection{Классификация методов}

\hspace{1.25cm}
Методы решения проблемы византийских генералов можно классифицировать на несколько категорий в зависимости от подходов к обеспечению консенсуса и требований к системе:

\begin{enumerate}

    \item \underline{Детерминированные алгоритмы}\\
    Эти методы основаны на чётко определённых правилах и последовательностях действий, например, алгоритмы, предполагающие гарантированную доставку сообщений. Примером является алгоритм Лэмпорта.

    \item \underline{Статистические методы}\\
    Используют вероятностный подход для достижения консенсуса, как, например, в алгоритмах, связанных с доказательством выполнения работы (\textit{Proof-of-Work, PoW}).

    \item \underline{Асинхронные протоколы}\\
    Методы, позволяющие работать в системах с неизвестными задержками, такие как алгоритмы византийской отказоустойчивости (BFT, \textit{Byzantine Fault Tolerance}).

    \item \underline{Гибридные подходы}\\
    Комбинируют детерминированные и статистические методы для достижения более высокой эффективности и отказоустойчивости. Эти подходы часто применяются в современных блокчейн-системах.

    \item \underline{Алгоритмы с дополнительными доверенными узлами}\\
    Включают использование специальных доверенных узлов (например, репутационных систем) для повышения безопасности.

\end{enumerate}

Каждая из этих категорий находит применение в зависимости от особенностей системы и её требований к производительности и безопасности.

\subsection{Описание методов}

\hspace{1.25cm}
В этом разделе будут рассмотрены основные алгоритмы, применяемые для решения задачи византийских генералов.


\subsubsection{Алгоритм Лэмпорта}

\hspace{1.25cm}
Решение задачи византийских генералов (ЗВГ) зависит от того какой тип имеют сообщения — устные или подписанные. Устными считаются такие сообщения, содержимое которых находится под абсолютным контролем отправителя. Далее приведено решение в случае устных сообщений (ОM -- Oral Messages).

\underline{Теорема}
\begin{enumerate}
    \item Если число лояльных генералов $n \leq 2m$, где $m$ — число предателей, то задача византийских генералов не имеет решения.
    \item Если $n \geq 2m + 1$, задача ЗВГ разрешима.
\end{enumerate}

Доказательство теоремы можно свести к случаю $n = 3$, где один из генералов является предателем.

\underline{Предположения для устных сообщений}
\begin{enumerate}
    \item[A1.] Каждое посланное сообщение доставляется неповреждённым.
    \item[A2.] Получателю сообщения известен его отправитель.
    \item[A3.] Отсутствие сообщения может быть обнаружено.
\end{enumerate}

Предполагается, что каждый генерал имеет возможность прямой связи с любым другим генералом. Если сообщение от генерала не поступило, по умолчанию считается, что его решение — «ОТСТУПАТЬ».

Алгоритм \(OM(m)\) определяется индуктивно по \(m \in \mathbb{N}\). Он задаёт правило, по которому каждый генерал передаёт своё решение другим генералам. Алгоритм основан на функции \texttt{majority}, которая определяется следующим образом:
\[
\texttt{majority}(v_1, \dots, v_{n-1}) = v_i \text{если большинство значений из } \{v_1, \dots, v_{n-1}\} \text{ равно } v_i
\]

Функция \texttt{majority} используется для достижения согласованности между генералами в условиях возможного наличия предателей.~\cite{vmath}

\underline{Алгоритм \(OM(0)\):}

\begin{enumerate}
    \item Генерал посылает своё решение другим генералам.
    \item Каждый генерал использует решение, переданное ему другим генералом, или принимает решение «ОТСТУПАТЬ», если сообщение не получено.
\end{enumerate}

\underline{Алгоритм \(OM(m)\), \(m > 0\):}

\begin{enumerate}
    \item Генерал посылает своё решение другим генералам.
    \item Каждый \(i\)-й генерал, получивший от другого генерала решение \(v_i\) (или, если решение не получено, считает его «ОТСТУПАТЬ»), высылает всем оставшимся \(n - 2\) генералам это решение, принятое по алгоритму \(OM(m-1)\).
    \item Для каждого \(i\) и для каждого \(j \neq i\), пусть \(v_j\) — это решение, которое генерал \(i\) получает от генерала \(j\) на этапе (2) (используя алгоритм \(OM(m-1)\)), или считает этим решением «ОТСТУПАТЬ», если решение не получено. Генерал \(i\) принимает решение:
    \[
    \texttt{majority}(v_1, \dots, v_{n-1}).
    \]
\end{enumerate}


\subsubsection{Алгоритм Proof of Work}

\hspace{1.25cm}
Главной идеей данного алгоритма является соглашение генералами выполнять заранее определенный протокол, который включает решение криптографической задачи. Решение этой задачи сложно для отправителя, но легко проверяется остальными участниками.

\underline{Пошаговый алгоритм:}

\begin{enumerate}

    \item \textbf{Формирование сообщения}\\
    Генерал создаёт план атаки и добавляет уникальный одноразовый номер (nonce):
    \[
    M = \text{plan} + \text{nonce}.
    \]

    \item \textbf{Решение криптографической задачи:}
    \begin{enumerate}
        \item Генерал вычисляет хэш сообщения $H(M)$.
        \item Проверяет, удовлетворяет ли $H(M)$ заданному условию (например, содержит $k$ начальных нулей).
        \item Если нет, увеличивает \text{nonce} и повторяет вычисления.
    \end{enumerate}

    \item \textbf{Отправка сообщения}\\
    Генерал передаёт сообщение $M$ остальным генералам.

    \item \textbf{Проверка}
    \begin{enumerate}
        \item Каждый генерал проверяет, соответствует ли $H(M)$ заданным требованиям.
        \item Если $H(M)$ некорректен, сообщение отвергается.
    \end{enumerate}

    \item \textbf{Принятие решения}
    \begin{enumerate}
        \item Если большинство генералов подтвердили сообщение, план принимается.
        \item Иначе сообщение считается недействительным.
    \end{enumerate}
\end{enumerate}

Система устойчива до тех пор, пока злоумышленники не захватят более 50\% вычислительных мощностей сети.~\cite{pogorelov}


\subsubsection{Алгоритм Proof of Stake}

\hspace{1.25cm}
Алгоритм Proof of Stake (PoS) можно представить как систему, где генералы принимают решения на основе зависит от процента их солдат в общей численности армии. Этот процесс позволяет получить консенсус, где вероятность принятия решения зависит от доли каждого генерала.

\underline{Пошаговый алгоритм:}

\begin{enumerate}
	\item \textbf{Формирование ставки} \ "Вес" каждого генерала определяется количеством солдат в его армии в зависимости от суммарного количества воинов. Эта доля будет влиять на его шанс быть выбранным для принятия решения или выдвижения нового плана.

	\item \textbf{Выбор генерала для выдвижения плана}\\
Выбор решения происходит случайным образом, но вероятность зависит от ставки, выдвинувших предложения по дальнейшим действиям генералов.

	\item \textbf{Предложение плана} \\
Генерал, выбранный для предложения решения, определяет атаковать или отступать. Этот план будет основой для дальнейших действий.

	\item \textbf{Проверка и одобрение плана} \\
Остальные генералы проверяют предложенный план и решают, поддержать его или нет. Каждый командир будет учитывать, насколько справедливым или правдоподобным кажется предложение, и, в зависимости от его ставки, может принять решение о поддержке.

	\item \textbf{Принятие решения} \\
Если большинство генералов (включая тех, кто имеет большую долю в войсках) поддерживают план, он принимается. Если же план не получает поддержку большинства, он отклоняется.
\end{enumerate}

Таким образом, вероятность принятия решения зависит от доли каждого генерала в общей армии: чем больше у генерала солдат, тем выше вероятность того, что его предложение будет принято.

Эта система также устойчива, пока предатели не захватят более 50\% доли в войсках.~\cite{forklog_PoS}


\subsubsection{Алгоритм Delegated Proof of Stake}

\hspace{1.25cm}
Алгоритм Delegated Proof of Stake (DPoS) можно представить как систему, где генералы достигают консенсуса путем делегирования полномочий наиболее доверенным из них. Этот процесс обеспечивает высокую производительность сети и одновременно поддерживает демократическое управление через голосование.

\underline{Пошаговый алгоритм:}

\begin{enumerate}

	\item \textbf{Выбор делегатов (свидетелей)}\\
Все генералы голосуют за доверенных кандидатов, которые представляют их интересы. Голоса взвешиваются в зависимости от доли солдат генерала в общей численности армии. Топ-N кандидатов становятся делегатами.

	\item \textbf{Создание блоков}\\
Делегаты поочередно выдвигают решения (атаковать или отступать), создавая "блоки" предложений. Очередность делегатов заранее определена.

	\item \textbf{Валидация решений}\\
Остальные делегаты проверяют предложения на корректность: если более 2/3 делегатов поддерживают предложение, оно принимается.

	\item \textbf{Обновление списка делегатов}\\
Если делегат работает неэффективно или действует во вред сети, он может быть смещён голосованием генералов. Это обеспечивает динамическую адаптацию к изменениям в составе армии.

\end{enumerate}

Таким образом, DPoS обеспечивает быстрое и эффективное принятие решений, сохраняя устойчивость системы, пока доверенные делегаты остаются лояльными. Эта модель позволяет обрабатывать большее количество транзакций в секунду (TPS) по сравнению с PoW и PoS.~\cite{binance_DPoS}

\newpage