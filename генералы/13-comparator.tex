% Оценка и сравнение алгоритмов
\section{Оценка и сравнение алгоритмов}

\subsection{Критерии сравнения методов}

\hspace{1.25cm}
Для сравнения алгоритмов решения задачи византийских генералов будут использоваться следующие критерии:

\begin{enumerate}
    \item \underline{Тип системы:} синхронная или асинхронная.
    \item \underline{Устойчивость к предателям:} минимальное количество лояльных узлов, необходимое для достижения консенсуса.
    \item \underline{Эффективность:} количество сообщений или вычислительных ресурсов, необходимых для работы алгоритма.
    \item \underline{Масштабируемость:} способность алгоритма функционировать с увеличением числа участников.
    \item \underline{Обеспечение безопасности:} устойчивость к различным видам атак.
    \item \underline{Область применения:} в каких типах распределённых систем используется алгоритм.
\end{enumerate}


\subsection{Сравнительный анализ}

\hspace{1.25cm}
По сформулированным выше критериям была составлена таблица рассмотренных методов решения проблемы византийских генералов (см таблицу \ref{table:compare}).

\begin{table}[h!]
\centering
\begin{tabular}{|m{2.6cm}|m{3cm}|m{3cm}|m{3cm}|m{3cm}|}
\hline
\textbf{Критерий} & \textbf{Алгоритм Лэмпорта} & \textbf{Proof of Work (PoW)} & \textbf{Proof of Stake (PoS)} & \textbf{Delegated Proof of Stake (DPoS)} \\ \hline
\textbf{Тип системы} & Синхронная & Асинхронная & Асинхронная & Асинхронная \\ \hline
\textbf{Устойчивость к предателям} & \(n \geq 2m + 1\) & \(> 50\%\) вычислительных мощностей честных узлов & \(> 50\%\) доли у честных узлов & \(> \frac{2}{3}\) голосов доверенных делегатов \\ \hline
\textbf{Эффектив- ность} & Высокая (при малом \(n\)) & Низкая (много вычислений) & Средняя & Высокая \\ \hline
\textbf{Масштабируе- мость} & Ограниченная \(n\) & Ограниченная скоростью вычислений & Высокая & Очень высокая \\ \hline
\textbf{Безопасность} & Высокая & Высокая, но уязвима к 51\%-атаке & Высокая (при распределённой доле) & Зависит от честности делегатов \\ \hline
\textbf{Область применения} & Надёжные синхронные сети & Криптовалюты, распределённые сети & Криптовалюты, DeFi & Высоконагружен- ные блокчейн-системы \\ \hline
\end{tabular}
\caption{Сравнение алгоритмов решения задачи византийских генералов.}
\label{table:compare}
\end{table}


\subsection*{Выводы}

\hspace{1.25cm}
Благодаря проведённому анализу можно сделать следующие выводы:

\begin{enumerate}
    \item Алгоритм Лэмпорта лучше подходит для небольших систем с гарантированной синхронностью.
    \item PoW обеспечивает высокий уровень безопасности, но требует больших вычислительных затрат, поэтому подходит для систем, где приоритет — децентрализация (например, Bitcoin).
    \item PoS улучшает энергоэффективность по сравнению с PoW, сохраняя безопасность, что делает его популярным выбором для современных криптовалют (например, Ethereum 2.0).
    \item DPoS оптимален для высоконагруженных систем, где важна производительность и масштабируемость, но требует доверия к делегатам.
\end{enumerate}

\newpage